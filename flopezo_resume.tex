%%%%%%%%%%%%%%%%%
% This is an sample CV template created using altacv.cls
% (v1.6.5, 3 Nov 2022) written by LianTze Lim (liantze@gmail.com). Compiles with pdfLaTeX, XeLaTeX and LuaLaTeX.
%
%% It may be distributed and/or modified under the
%% conditions of the LaTeX Project Public License, either version 1.3
%% of this license or (at your option) any later version.
%% The latest version of this license is in
%%    http://www.latex-project.org/lppl.txt
%% and version 1.3 or later is part of all distributions of LaTeX
%% version 2003/12/01 or later.
%%%%%%%%%%%%%%%%

%% Use the "normalphoto" option if you want a normal photo instead of cropped to a circle
% \documentclass[10pt,a4paper,normalphoto]{altacv}

\documentclass[10pt,a4paper,ragged2e,withhyper]{altacv}
%% AltaCV uses the fontawesome5 and packages.
%% See http://texdoc.net/pkg/fontawesome5 for full list of symbols.

% Change the page layout if you need to
% \geometry{left=1.25cm, right=1.25cm, top=1.5cm, bottom=1.5cm, columnsep=1.2cm}
\geometry{left=1.4cm, top=.8cm, right=1.4cm, bottom=1.8cm, footskip=.5cm, columnsep=1.2cm}

% The paracol package lets you typeset columns of text in parallel
\usepackage{paracol}

% Change the font if you want to, depending on whether
% you're using pdflatex or xelatex/lualatex
% \ifxetexorluatex
%   % If using xelatex or lualatex:
%   \setmainfont{Roboto Slab}
%   \setsansfont{Lato}
%   \renewcommand{\familydefault}{\sfdefault}
% \else
%   % If using pdflatex:
%   \usepackage[rm,light]{roboto}
%   \usepackage[defaultsans]{lato}
%   \usepackage{sourcesanspro}

%   \renewcommand{\familydefault}{\sfdefault}
% \fi

% Change the colours if you want to
\definecolor{SlateGrey}{HTML}{2E2E2E}
\definecolor{LightGrey}{HTML}{666666}
\definecolor{DarkPastelRed}{HTML}{450808}
\definecolor{PastelRed}{HTML}{8F0D0D}
\definecolor{GoldenEarth}{HTML}{E7D192}
\definecolor{ModerateBlue}{HTML}{5d7bc6}
\definecolor{DarkGrey}{HTML}{2f4f4f}
\colorlet{name}{ModerateBlue}
\colorlet{tagline}{DarkGrey}
\colorlet{heading}{ModerateBlue}
\colorlet{headingrule}{DarkGrey}
\colorlet{subheading}{DarkGrey}
\colorlet{accent}{DarkGrey}
\colorlet{emphasis}{DarkGrey}
\colorlet{body}{DarkGrey}

% Change some fonts, if necessary
% \renewcommand{\namefont}{\Huge\rmfamily\bfseries}
% \renewcommand{\personalinfofont}{\normalsize}
% \renewcommand{\cvsectionfont}{\LARGE\rmfamily\bfseries}
% \renewcommand{\cvsubsectionfont}{\large\bfseries}

\usepackage{roboto}
% \renewcommand*{\rmdefault}{roboto}

\renewcommand{\namefont}{\Huge\rmfamily\bfseries}
\renewcommand{\personalinfofont}{\normalsize\robotolight}
\renewcommand{\cvsectionfont}{\LARGE\rmfamily\bfseries}
\renewcommand{\cvsubsectionfont}{\large\bfseries}


% Change the bullets for itemize and rating marker
% for \cvskill if you want to
\renewcommand{\itemmarker}{{\small\textbullet}}
\renewcommand{\ratingmarker}{\faCircle}

%% Use (and optionally edit if necessary) this .tex if you
%% want to use an author-year reference style like APA(6)
%% for your publication list
% When using APA6 if you need more author names to be listed
% because you're e.g. the 12th author, add apamaxprtauth=12
\usepackage[backend=biber,style=apa6,sorting=ydnt]{biblatex}
\defbibheading{pubtype}{\cvsubsection{#1}}
\renewcommand{\bibsetup}{\vspace*{-\baselineskip}}
\AtEveryBibitem{%
  \makebox[\bibhang][l]{\itemmarker}%
  \iffieldundef{doi}{}{\clearfield{url}}%
}
\setlength{\bibitemsep}{0.25\baselineskip}
\setlength{\bibhang}{1.25em}


%% Use (and optionally edit if necessary) this .tex if you
%% want an originally numerical reference style like IEEE
%% for your publication list
% \usepackage[backend=biber,style=ieee,sorting=ydnt]{biblatex}
%% For removing numbering entirely when using a numeric style
\setlength{\bibhang}{1.25em}
\DeclareFieldFormat{labelnumberwidth}{\makebox[\bibhang][l]{\itemmarker}}
\setlength{\biblabelsep}{0pt}
\defbibheading{pubtype}{\cvsubsection{#1}}
\renewcommand{\bibsetup}{\vspace*{-\baselineskip}}
\AtEveryBibitem{%
  \iffieldundef{doi}{}{\clearfield{url}}%
}


%% sample.bib contains your publications
\addbibresource{publications.bib}

% Suppress hyphens
\tolerance=1
\emergencystretch=\maxdimen
\hyphenpenalty=10000
\hbadness=10000

\NewInfoField{googlescholar}{\faGraduationCap}[https://scholar.google.com/citations?user=]

\begin{document}
\name{Federico L\'opez-Osorio}
\tagline{Bioinformatics | Genomics}
%% You can add multiple photos on the left or right
% \photoR{2.8cm}{Globe_High}
% \photoL{2.5cm}{Yacht_High,Suitcase_High}

\personalinfo{%
  % Not all of these are required!
  % \mailaddress{Hermannstraße 4, 06108}
  % \phone{+49 160 3268243}
  \phone{+00 000 0000000}
  % \email{flopezo84@gmail.com}
  \email{f********@gmail.com}
  % \location{Halle (Saale), Germany}
  %\homepage{}
  \github{flopezo}
  \googlescholar{zOAvFzMAAAAJ}
  \orcid{0000-0001-7654-7967}
  \linkedin{federico-lopez-osorio-a46b5754}
  \twitter{@flopezos}
  %% You can add your own arbitrary detail with
  %% \printinfo{symbol}{detail}[optional hyperlink prefix]
  % \printinfo{\faPaw}{Hey ho!}[https://example.com/]
  %% Or you can declare your own field with
  %% \NewInfoFiled{fieldname}{symbol}[optional hyperlink prefix] and use it:
  % \NewInfoField{gitlab}{\faGitlab}[https://gitlab.com/]
  % \gitlab{your_id}
  %%
  %% For services and platforms like Mastodon where there isn't a
  %% straightforward relation between the user ID/nickname and the hyperlink,
  %% you can use \printinfo directly e.g.
  % \printinfo{\faMastodon}{@username@instace}[https://instance.url/@username]
  %% But if you absolutely want to create new dedicated info fields for
  %% such platforms, then use \NewInfoField* with a star:
  % \NewInfoField*{mastodon}{\faMastodon}
  %% then you can use \mastodon, with TWO arguments where the 2nd argument is
  %% the full hyperlink.
  % \mastodon{@username@instance}{https://instance.url/@username}
}

\makecvheader
%% Depending on your tastes, you may want to make fonts of itemize environments slightly smaller
% \AtBeginEnvironment{itemize}{\small}

%% Set the left/right column width ratio to 6:4.
\columnratio{0.6}

% Start a 2-column paracol. Both the left and right columns will automatically
% break across pages if things get too long.
\begin{paracol}{2}
\cvsection{\color{ModerateBlue}\faAddressCard Summary}
I am a biologist with expertise in bioinformatics, genomic analysis, and modern 
techniques to visualise data. My scientific research focuses on discovering the 
genetic features underlying traits of interest in insects and other organisms.
I integrate genome-wide data using reproducible workflows and communicate
findings to the scientific community, the general public, and stakeholders.

\cvsection{\color{ModerateBlue}\faBusinessTime Experience}

\cvevent{Marie Skłodowska-Curie Actions Fellow}{Queen Mary University of London}{2019 -- 2022}{London, UK}
\begin{itemize}
\item Analysed population genomic data to understand the origin of a social supergene in ants
\item Co-authored a responsive mode research grant proposal successfully funded by 
the Biotechnology and Biological Sciences Research Council (BBSRC) (ref. BB/T015683/1)
\item Published research papers in high-impact journals
\item Actively participated in research projects of graduate students and contributed to their mentoring
\end{itemize}

\divider

\cvevent{Postdoctoral researcher}{Queen Mary University of London}{2018 -- 2019}{London, UK}
\begin{itemize}
\item Analysed hundreds of RNA-seq samples of bees to investigate the effects of exposure to various insecticides
\item Wrote shell and R scripts to identify differentially expressed genes and visualise results
\item Disseminated findings in poster format at an international conference
\item Assisted graduate students in their research projects 
\end{itemize}

\divider

\cvevent{Gerstner Scholars Program (Biology) Fellow}{American Museum of Natural History}{2016 -- 2018}{New York, NY, USA}
\begin{itemize}
\item Designed, managed, and developed a comparative genomics study of social wasps
\item Collected field samples and sequenced genomes and transcriptomes of wasp species
\item Developed bioinformatics workflows for \textit{de novo} assembly of genomes and transcriptomes,
annotating genomes, finding differentially expressed genes, orthology inference, and phylogenomics 
\item Presented findings in talks at international conferences
\item Wrote progress and final reports
\end{itemize}

% \cvsection{Projects}

% \cvevent{Project 1}{Funding agency/institution}{}{}
% \begin{itemize}
% \item Details
% \end{itemize}

% \divider

% \cvevent{Project 2}{Funding agency/institution}{Project duration}{}
% A short abstract would also work.

% \medskip

% \cvsection{Bioinformatics competencies}

% % Adapted from @Jake's answer from http://tex.stackexchange.com/a/82729/226
% % \wheelchart{outer radius}{inner radius}{
% % comma-separated list of value/text width/color/detail}
% \wheelchart{1.5cm}{1.0cm}{%
%   25/10em/accent!40/{Genomics: genome and transcriptome assembly, predicting gene structures...},
%   15/10em/accent!30/{Metagenomics: taxonomic labeling of DNA sequences},
%   20/14em/accent!60/{RNA-seq: detecting differentially expressed genes},
%   15/8em/accent!20/{Population genomics: identifying SNPs associated with phenotypes...},
%   25/8em/accent/{Phylogenomics: orthology inference, phylogenetic analysis}
% }

% use ONLY \newpage if you want to force a page break for
% ONLY the current column
\newpage

% \cvsection{\color{ModerateBlue}\faFile* Publications}
\cvsection{\color{ModerateBlue}\faFile Publications}

%% Specify your last name(s) and first name(s) as given in the .bib to automatically bold your own name in the publications list.
%% One caveat: You need to write \bibnamedelima where there's a space in your name for this to work properly; or write \bibnamedelimi if you use initials in the .bib
%% You can specify multiple names, especially if you have changed your name or if you need to highlight multiple authors.
\mynames{López-Osorio/Federico}
%% MAKE SURE THERE IS NO SPACE AFTER THE FINAL NAME IN YOUR \mynames LIST

\nocite{*}

\printbibliography[heading=pubtype,title={\printinfo{\faFile*[regular]}{Journal Articles}},type=article]

\divider

\printbibliography[heading=pubtype,title={\printinfo{\faBook}{Book Chapters}},type=book]

\divider

\printbibliography[heading=pubtype,title={\printinfo{\faUsers}{Conference Proceedings}},type=inproceedings]

%% Switch to the right column. This will now automatically move to the second
%% page if the content is too long.
\switchcolumn

% \cvsection{}

% \begin{quote}
% ``''
% \end{quote}

\cvsection{\color{ModerateBlue}\faUniversity Education}

\cvevent{Ph.D.\ in Biology}{University of Vermont}{Aug. 2010 -- May 2016}{}
Dissertation: Phylogenetics and molecular evolution of highly eusocial wasps

\divider

\cvevent{B.Sc.\ in Biology}{Universidad Industrial de Santander}{2009}{}
% Thesis: A phylogenetic approach to conserving Amazonian biodiversity

\cvsection{\color{ModerateBlue}\faCog Skills}

% \setlist[itemize]{noitemsep, topsep=0pt}
\setlist{leftmargin=-4mm, noitemsep, topsep=0pt}

\cvachievement{\raisebox{2.0\height}{\color{ModerateBlue}\faLaptopCode}}{Bioinformatics}{
\begin{itemize}%[label={}]
\setlength\itemsep{0em}
\item Analysis of high-throughput genomic data
\item Quality control of genome-scale assays
\item Genome and transcriptome assembly, including non-model organisms
\item Structural and functional genome  annotation
\item Orthology inference and phylogenomics
\item Differential gene expression and transcript usage (RNA-seq)
\item Variant calling and analysis
\item Metagenome assembly and taxonomic profiling
\end{itemize}
}\vspace*{-\baselineskip}

\divider

\cvachievement{\raisebox{2.0\height}{\color{ModerateBlue}\faChartArea}}{Data Analysis}{
\begin{itemize}%[label={}]
\setlength\itemsep{0em}
\item Data visualisation (e.g., ggplot2)
\item Linear models (e.g., GLM, GLMM)
\item Data normalisation and transformation
\item Unsupervised learning (e.g., PCA)
\item Experimental design
\item Effective reporting with aesthetic appeal
\end{itemize}}\vspace*{-\baselineskip}

\divider

\cvachievement{\raisebox{2.0\height}{\color{ModerateBlue}\faCode}}{Programming}{
% \cvtag{R} \cvtag{Shell} \cvtag{Python}
\begin{itemize}%[label={}]
\setlength\itemsep{0em}
\item R, Tidyverse, Shiny
\item Shell scripting
\item Unix systems, high-performance computing
\item git, GitHub
\item LaTeX, R Markdown, Quarto
\item Workflow management (e.g., Snakemake)
\item Package management (e.g., Conda)
\item Python
\end{itemize}}\vspace*{-\baselineskip}

% \cvsection{Skills}

% \cvtag{Hard-working}
% \cvtag{Eye for detail}\\
% \cvtag{Motivator \& Leader}

% \divider\smallskip

% \cvtag{C++}
% \cvtag{Embedded Systems}\\
% \cvtag{Statistical Analysis}

\cvsection{\color{ModerateBlue}\faLanguage Languages}

English (fluent) | Spanish (native)

% \cvskill{English}{5}
% \divider

% \cvskill{Spanish}{5}
% \divider

% \cvskill{German}{3.5} %% Supports X.5 values.

%% Yeah I didn't spend too much time making all the
%% spacing consistent... sorry. Use \smallskip, \medskip,
%% \bigskip, \vspace etc to make adjustments.
% \medskip

% \divider

% \cvsection{References}

% % \cvref{name}{email}{mailing address}
% \cvref{Prof.\ Alpha Beta}{Institute}{a.beta@university.edu}
% {Address Line 1\\Address line 2}

% \divider

% \cvref{Prof.\ Gamma Delta}{Institute}{g.delta@university.edu}
% {Address Line 1\\Address line 2}


\end{paracol}


\end{document}
